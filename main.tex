% !Mode:: "TeX:UTF-8"
% !TEX builder = LATEXMK
% !TEX program = xelatex

% TODO disable draft mode and enable anon if necessary
\documentclass[twoside,copyright,AutoFakeBold]{style/gdoc}
% \documentclass[anon,master,twoside,nocpsupervisor]{style/zjuthesis}

% 插图路径设置,图片放在figures 文件夹下。一般来说论文的插图比较多,通常按章节存
% 放,因此可以在以下命令中在按章节添加存放图片的文件夹路径。如以下这个路径中 ./
% 代表当前main.tex所在的目录,就是一般所说的当前文件夹;figures 文件夹就是子文件
% 夹,存放正文及附录中要用到的所有的图片,在figures 文件夹中的子文件夹就是存放各
% 个章节图片的文件夹,一般命名与相应章节的名字相同,如intro 章节用到的图片全放在
% 了intro 这个子文件夹下。
% \graphicspath{%
%     {./figures/intro/}%
% }

% 一级标题
\title{通用文档\LaTeX{}模板}
% 二级标题
\secondarytitle{适用于一般文档撰写}
% 文档编号
\docnumber{}
% 文档版本
\docversion{1.0.0}
% 文档编制人
\author{Anonymous}
% 文档审核人
\docreviewer{Anonymous}
% 文档批准人
\docapprover{Anonymous}
% 提交日期
\submitdate{\today}
% 文档联系人
\doccontact{Anonymous}
% 文档联系邮箱
\docemail{anonymous@example.com}
% 文档所属机构
\docinstitute{浙江大学}
% 文档所属机构
\docenglishinstitute{Zhejiang University}

\begin{document}

\maketitle

% !Mode:: "TeX:UTF-8"
% !TEX root = ../main.tex

\chapter*{文档版本管理}
\label{cha:version_control}

\begin{quote}
  \kaishu
  \textbf{版本号说明:}本文档采用的版本号格式为a.b.c,其中a代表重大的版本变迁,b代表在a版本下的重要内容更新,c代表关键性的bug修复。
\end{quote}

\begin{table}[ht]
  \zihao{-5}
  \centering
  \caption{文档版本记录}
  \label{tab:version_control}
  \begin{tabular}{p{1cm}|p{1.5cm}|p{1.0cm}|p{1.5cm}|p{1.5cm}|p{1.5cm}|p{1.5cm}|p{3cm}}
    \hline\hline
    编号       & 文件状态 & 版本   &
    修改人     & 审核人   & 批准人 &
    修改日期   & 备注 \\
    \hline
    1          & 新建     & 1.0.0  &
    Anonymous  &          &        &
    2020.01.08 & 创建文档架构、更新文档初稿 \\
    \hline\hline
  \end{tabular}
\end{table}
% chapter 文档版本管理 (end)


\frontmatter

% 正文目录:
\tableofcontents
% 插图目录:
% \listoffigures
% 表格目录:
% \listoftables
% \include{contents/denotation}

\mainmatter

% !Mode:: "TeX:UTF-8"
% !TEX root = ../thesis.tex

\chapter{简介}
\label{cha:introduction}

% chapter 简介 (end)


\backmatter

% \bibliography{main}
% \nocite{*} % to show the entire references, annotate it if need.

% \appendix
% \include{thesis/appendixA}
% \include{thesis/appendixB}
\end{document}
